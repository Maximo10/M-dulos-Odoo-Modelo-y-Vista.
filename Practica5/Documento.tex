\documentclass[12pt,a4paper]{article}

\usepackage[spanish]{babel}
\usepackage[utf8]{inputenc}
\usepackage{float}
\usepackage{hyperref}
\hypersetup{
    colorlinks=true,
    linkcolor=blue,
    urlcolor=cyan,
    pdftitle={Documentación Práctica 4 Odoo},
    pdfauthor={Birhan Fernández},
    bookmarks=true,
    pdfpagemode=FullScreen,
}
\usepackage{geometry}
\geometry{margin=2.5cm}
\usepackage{xcolor}
\usepackage{graphicx}
\usepackage{setspace}
\usepackage{tcolorbox}
\tcbuselibrary{listings,skins,breakable}

\onehalfspacing 

\newtcblisting{codeblock}[2][]{
    listing only,
    listing options={language=,breaklines=true,basicstyle=\ttfamily\small},
    colback=gray!10,
    colframe=gray!70,
    boxrule=0.5mm,
    arc=2mm,
    breakable,
    enhanced,
    #1
}

\newcommand{\volverindice}{\noindent\hyperlink{toc}{\textcolor{blue}{\textbf{Volver al índice}}}\par\vspace{0.5cm}}

\begin{document}

\begin{titlepage}
    \centering
    \vspace*{3cm}
    {\Huge\bfseries Práctica 4: Módulos Odoo – Modelos y Vistas \par}
    \vspace{2cm}
    {\Large Autor: Birhan Fernández \par}
    {\Large Fecha de entrega: 15 de diciembre \par}
    \vfill
    \includegraphics[width=0.7\textwidth]{odoo_logo.png} % Reemplaza con tu logo si tienes
\end{titlepage}

\tableofcontents
\newpage

\section{Introducción}

En este documento se presenta una guía detallada sobre el desarrollo de las actividades
propuestas en la Práctica 4 del módulo de Odoo.

El objetivo principal de esta práctica es aplicar los conocimientos adquiridos sobre
modelos de datos, relaciones y vistas en el desarrollo de aplicaciones dentro del entorno
Odoo. Para ello, se han implementado distintos módulos que abarcan diferentes contextos,
como la gestión de tareas, una biblioteca de cómics, la gestión de pacientes y médicos en
un hospital y la organización de ciclos formativos en un instituto.

Se pretende que el usuario entienda no solo la estructura técnica de cada módulo, sino
también cómo interactúan los modelos, las relaciones y las vistas, y cómo estas se
integran en la interfaz de Odoo para facilitar la gestión de información.

\section{Consideraciones previas}

Antes de comenzar con el desarrollo de las actividades, es necesario realizar una serie
de pasos previos. En primer lugar, el usuario debe acceder al repositorio de GitHub donde
se encuentran los módulos de ejemplo proporcionados:

\texttt{https://github.com/sergarb1/OdooModulosEjemplos}

\textbf{[IMG]}

Una vez dentro del repositorio, se procede a descargar los ejemplos en formato \texttt{.ZIP}.
Tras finalizar la descarga, los módulos se copian dentro de la carpeta \texttt{addons} del
entorno Odoo utilizado.

\textbf{[IMG]}

Con los módulos añadidos al sistema, se realiza una actualización de la lista de módulos
desde Odoo utilizando permisos de administrador. Posteriormente, desde el buscador de
aplicaciones, se introduce el prefijo \texttt{EJ0} para comprobar que los módulos se han
instalado correctamente.

\textbf{[IMG]}

Una vez verificado que los módulos funcionan correctamente, se puede dar inicio al
desarrollo de las actividades propuestas en la práctica.

\section{Actividad 01: Modificación de Lista de Tareas}

Esta actividad consiste en añadir nuevas vistas al módulo de lista de tareas desarrollado
en prácticas anteriores. Se busca que el usuario pueda visualizar las tareas en formato
Kanban y Calendario.

\subsection{Añadir vista Kanban}

Para añadir la vista Kanban, se debe editar el archivo \texttt{views.xml} dentro del
módulo y agregar el siguiente bloque:

\begin{codeblock}
<!-- ==================================================
     Vista kanban (kanban)
===================================================
Define la vista Kanban -->
<record model="ir.ui.view" id="lista_tareas_kanban">
    <field name="name">lista_tareas.kanban</field>
    <field name="model">lista_tareas.lista_tareas</field>
    <field name="arch" type="xml">
        <kanban>
            <field name="tarea"/>
            <field name="prioridad"/>
            <field name="urgente"/>
            <templates>
                <t t-name="kanban-box">
                    <div class="oe_kanban_global_click">
                        <strong><field name="tarea"/></strong>
                        <div>Prioridad: <field name="prioridad"/></div>
                        <div t-if="record.urgente.value">Urgente</div>
                    </div>
                </t>
            </templates>
        </kanban>
    </field>
</record>
\end{codeblock}

\subsection{Acción y menú de la vista Kanban}

A continuación, se define una acción para acceder a la vista y se añade un menú en el
módulo:

\begin{codeblock}
<!-- Acción principal -->
<record model="ir.actions.act_window" id="lista_tareas_action_window">
    <field name="name">Listado de tareas pendientes</field>
    <field name="res_model">lista_tareas.lista_tareas</field>
    <field name="view_mode">list,kanban</field>
</record>

<!-- Menú -->
<menuitem name="Listado de tareas" id="lista_tareas_menu_root"/>
<menuitem name="Opciones Lista Tareas" id="lista_tareas_menu_1" parent="lista_tareas_menu_root"/>
<menuitem name="Mostrar lista" id="lista_tareas_menu_1_list" parent="lista_tareas_menu_1" action="lista_tareas_action_window"/>
\end{codeblock}

\subsection{Añadir vista Calendario}

Se añade un campo de fecha en el modelo de tareas para luego generar la vista Calendario:

\begin{codeblock}
# models/models.py
fecha_vencimiento = fields.Date(string="Fecha de Vencimiento")
\end{codeblock}

Luego, se crea la vista Calendario y se actualiza la acción principal:

\begin{codeblock}
<!-- Vista Calendario -->
<record id="lista_tareas_calendar_view" model="ir.ui.view">
    <field name="name">lista_tareas.calendar</field>
    <field name="model">lista_tareas.lista_tareas</field>
    <field name="arch" type="xml">
        <calendar date_start="fecha_vencimiento" string="Tareas por fecha">
            <field name="tarea"/>
            <field name="prioridad"/>
        </calendar>
    </field>
</record>

<!-- Actualizar acción principal -->
<field name="view_mode">list,form,kanban,calendar</field>
\end{codeblock}


\section{Actividad 02: Biblioteca de Cómics}

En esta actividad se implementa la gestión de socios y el control de préstamos de cómics.

\subsection{Modelo Socio}

Se crea una clase \texttt{Socio} para almacenar los datos de los socios:

\begin{codeblock}
class Socio(models.Model):
    _name = 'biblioteca.socio'
    _description = 'Socio de la biblioteca'
    _rec_name = 'identificador'

    nombre = fields.Char('Nombre', required=True)
    apellido = fields.Char('Apellido', required=True)
    identificador = fields.Char('Identificador', required=True, help='ID único del socio')
\end{codeblock}

\subsection{Modelo Ejemplar}

Se crea la clase \texttt{Ejemplar} para gestionar los préstamos:

\begin{codeblock}
class Ejemplar(models.Model):
    _name = 'biblioteca.ejemplar'
    _description = 'Ejemplar de comic para préstamo'
    _rec_name = 'comic_id'

    comic_id = fields.Many2one('biblioteca.comic', string='Comic', required=True)
    prestado_a = fields.Many2one('biblioteca.socio', string='Prestado a')
    fecha_prestamo = fields.Date('Fecha de préstamo')
    fecha_devolucion = fields.Date('Fecha prevista de devolución')

    estado = fields.Selection(
        [('disponible', 'Disponible'),
         ('prestado', 'Prestado')],
        string='Estado',
        compute='_compute_estado',
        store=True
    )
        @api.depends('prestado_a', 'fecha_prestamo', 'fecha_devolucion')
    def _compute_estado(self):
        for rec in self:
            if rec.prestado_a and rec.fecha_prestamo and rec.fecha_devolucion:
                rec.estado = 'prestado'
            else:
                rec.estado = 'disponible'

    @api.constrains('fecha_prestamo')
    def _check_fecha_prestamo(self):
        today = fields.Date.today()
        for record in self:
            if record.fecha_prestamo and record.fecha_prestamo > today:
                raise ValidationError('La fecha de préstamo no puede ser posterior a hoy.')

    @api.constrains('fecha_devolucion')
    def _check_fecha_devolucion(self):
        today = fields.Date.today()
        for record in self:
            if record.fecha_devolucion and record.fecha_devolucion < today:
                raise ValidationError('La fecha de devolución no puede ser anterior a hoy.')

    @api.onchange('prestado_a')
    def _onchange_prestado_a(self):
        if self.prestado_a:
            self.estado = 'prestado'
        else:
            self.estado = 'disponible'
\end{codeblock}

\subsection{Vistas y Menús}

Se definen las acciones, menús y vistas para los cómics, socios y ejemplares. Cada menú
permite gestionar de forma clara cada tipo de información.

\textbf{Menús:}
\begin{itemize}
    \item Gestión de cómics
    \item Gestión de socios
    \item Gestión de ejemplares
\end{itemize}
\newpage
\textbf{Gestión de cómics:}
\begin{codeblock}
        <!-- ========================= 
        ACCIONES Y MENÚS COMICS 
        ========================= -->
    <record id='biblioteca_comic_action' model='ir.actions.act_window'>
        <field name="name">Biblioteca de Comics</field>
        <field name="res_model">biblioteca.comic</field>
        <field name="view_mode">list,form</field>
    </record>

    <menuitem name="Mi biblioteca (Simple)" id="biblioteca_base_menu"/>
    <menuitem name="Comics" id="biblioteca_comic_menu" parent="biblioteca_base_menu" action="biblioteca_comic_action"/>

    <!-- ========================= 
        VISTAS COMICS 
        ========================= -->
    <record id="biblioteca_comic_view_form" model="ir.ui.view">
        <field name="name">Formulario Comic</field>
        <field name="model">biblioteca.comic</field>
        <field name="arch" type="xml">
            <form>
                <header>
                    <button type="object" name="archivar" string="Archivar Comics"/>
                </header>
                <group>
                    <group>
                        <field name="nombre"/>
                        <field name="autor_ids" widget="many2many_tags"/>
                        <field name="estado"/>
                        <field name="paginas"/>
                        <field name="activo" readonly="1"/>
                    </group>
                    <group>
                        <field name="precio"/>
                        <field name="fecha_publicacion"/>
                        <field name="portada" widget="image" class="oe_avatar"/>
                        <field name="valoracion_lector"/>
                    </group>
                </group>
                <group>
                    <field name="descripcion"/>
                </group>
            </form>
        </field>
    </record>

    <record id="biblioteca_comic_view_list" model="ir.ui.view">
        <field name="name">Lista Comics</field>
        <field name="model">biblioteca.comic</field>
        <field name="arch" type="xml">
            <list>
                <field name="nombre"/>
                <field name="fecha_publicacion"/>
                <field name="estado"/>
            </list>
        </field>
    </record>
\end{codeblock}
\newpage
\textbf{Gestión de socios:}
\begin{codeblock}
 <!-- ========================= 
     ACCIONES Y MENÚS SOCIOS 
     ========================= -->
    <record id="action_socio" model="ir.actions.act_window">
        <field name="name">Socios</field>
        <field name="res_model">biblioteca.socio</field>
        <field name="view_mode">list,form</field>
    </record>

    <menuitem id="menu_socio" name="Socios" parent="biblioteca_base_menu" action="action_socio"/>

    <!-- ========================= -->
    <!-- VISTAS SOCIOS -->
    <!-- ========================= -->
    <record id="view_socio_list" model="ir.ui.view">
        <field name="name">Lista Socios</field>
        <field name="model">biblioteca.socio</field>
        <field name="arch" type="xml">
            <list>
                <field name="identificador"/>
                <field name="nombre"/>
                <field name="apellido"/>
            </list>
        </field>
    </record>

    <record id="view_socio_form" model="ir.ui.view">
        <field name="name">Formulario Socio</field>
        <field name="model">biblioteca.socio</field>
        <field name="arch" type="xml">
            <form>
                <group>
                    <field name="identificador"/>
                    <field name="nombre"/>
                    <field name="apellido"/>
                </group>
            </form>
        </field>
    </record>
\end{codeblock}


\textbf{Gestión de ejemplares:}
\begin{codeblock}
 <!-- ========================= -->
    <!-- ACCIONES Y MENÚS EJEMPLARES -->
    <!-- ========================= -->
    <record id="action_ejemplar" model="ir.actions.act_window">
        <field name="name">Ejemplares</field>
        <field name="res_model">biblioteca.ejemplar</field>
        <field name="view_mode">list,form</field>
    </record>

    <menuitem id="menu_ejemplar" name="Ejemplares" parent="biblioteca_base_menu" action="action_ejemplar"/>

    <!-- ========================= 
        VISTAS EJEMPLARES 
        ========================= -->
    <record id="view_ejemplar_list" model="ir.ui.view">
        <field name="name">Lista Ejemplares</field>
        <field name="model">biblioteca.ejemplar</field>
        <field name="arch" type="xml">
            <list>
                <field name="comic_id"/>
                <field name="prestado_a"/>
                <field name="fecha_prestamo"/>
                <field name="fecha_devolucion"/>
                <field name="estado"/>
            </list>
        </field>
    </record>

    <record id="view_ejemplar_form" model="ir.ui.view">
        <field name="name">Formulario Ejemplar</field>
        <field name="model">biblioteca.ejemplar</field>
        <field name="arch" type="xml">
            <form>
                <group>
                    <field name="comic_id"/>
                    <field name="prestado_a"/>
                    <field name="fecha_prestamo"/>
                    <field name="fecha_devolucion"/>
                    <field name="estado" readonly="1"/>
                </group>
            </form>
        </field>
    </record>
\end{codeblock}


\section{Actividad 03: Gestión Hospitalaria}

En esta actividad se desarrolla un módulo para la gestión básica de un hospital, que
permite administrar pacientes, médicos y consultas. El objetivo es aplicar relaciones
\texttt{Many2one} y validar información en Odoo.

\subsection{Modelo Paciente}

El modelo \texttt{Paciente} almacena los datos de cada paciente y sus síntomas:

\begin{codeblock}
class Paciente(models.Model):
    _name='hospital.paciente'
    _description='Paciente del hospital'
    _rec_name='nombre'

    nombre = fields.Char('Nombre', required=True)
    apellidos = fields.Char('Apellidos', required=True)
    sintomas = fields.Text('Síntomas')
\end{codeblock}

\subsection{Modelo Médico}

El modelo \texttt{Medico} permite almacenar datos de los médicos del hospital:

\begin{codeblock}
class Medico(models.Model):
    _name='hospital.medico'
    _description='Médico del hospital'
    _rec_name='nombre'

    nombre = fields.Char('Nombre', required=True)
    apellidos = fields.Char('Apellidos', required=True)
    numero_colegiado = fields.Char('Número de colegiado', required=True,
                                   help='Identificador único del médico')
\end{codeblock}

\subsection{Modelo Consulta}

El modelo \texttt{Consulta} relaciona pacientes con médicos y almacena diagnósticos:

\begin{codeblock}
class Consulta(models.Model):
    _name='hospital.consulta'
    _description='Consulta entre paciente y médico'
    _rec_name='paciente_id'

    paciente_id = fields.Many2one('hospital.paciente', string='Paciente', required=True)
    medico_id = fields.Many2one('hospital.medico', string='Médico', required=True)
    diagnostico = fields.Text('Diagnóstico')
    fecha_consulta = fields.Date('Fecha de la consulta', default=fields.Date.today)

    @api.constrains('fecha_consulta')
    def _check_fecha_consulta(self):
        for record in self:
            if record.fecha_consulta and record.fecha_consulta > fields.Date.today():
                raise ValidationError('La fecha de la consulta no puede ser posterior a hoy.')
\end{codeblock}

\subsection{Vistas y Acciones}

Se definen vistas de lista y formulario para cada modelo, permitiendo gestionar la
información de forma clara desde la interfaz de Odoo.

\begin{codeblock}
<!-- Vista Lista Pacientes -->
<record id="view_paciente_list" model="ir.ui.view">
    <field name="name">hospital.paciente.list</field>
    <field name="model">hospital.paciente</field>
    <field name="arch" type="xml">
        <list>
            <field name="nombre"/>
            <field name="apellidos"/>
            <field name="sintomas"/>
        </list>
    </field>
</record>

<!-- Vista Formulario Pacientes -->
<record id="view_paciente_form" model="ir.ui.view">
    <field name="name">hospital.paciente.form</field>
    <field name="model">hospital.paciente</field>
    <field name="arch" type="xml">
        <form>
            <group>
                <field name="nombre"/>
                <field name="apellidos"/>
            </group>
            <group>
                <field name="sintomas"/>
            </group>
        </form>
    </field>
</record>
\end{codeblock}

De manera similar, se crean vistas para médicos y consultas. Las acciones permiten
acceder a estas vistas desde el menú principal:

\begin{codeblock}
<record id="action_pacientes" model="ir.actions.act_window">
    <field name="name">Pacientes</field>
    <field name="res_model">hospital.paciente</field>
    <field name="view_mode">list,form</field>
</record>

<menuitem id="menu_hospital_root" name="Hospital"/>
<menuitem id="menu_hospital_pacientes" name="Pacientes" parent="menu_hospital_root" action="action_pacientes"/>
\end{codeblock}

\section{Actividad 04: Gestión de Ciclos Formativos}

En esta actividad se desarrolla un módulo para la gestión de ciclos formativos,
módulos, alumnos y profesores en un instituto. Se aplican relaciones \textbf{One2many}
y \textbf{Many2many} para modelar adecuadamente la información.

\subsection{Modelo Ciclo Formativo}

\begin{codeblock}
#Definimos modelo Ciclo Formativo
class Cicloformativos(models.Model):
    #Nombre y descripcion del modelo
    _name='ciclo.formativo'
    _description='Ciclo Formativo'
    _rec_name='nombre'

    #Atributos
    nombre = fields.Char('Nombre', required=True)
    modulo_ids=fields.One2many('modulo', 'ciclo_id', string='Módulos')
\end{codeblock}

\subsection{Modelo Módulo}
\begin{codeblock}
#Definimos modelo Módulo
class Modulo(models.Model):
    #Nombre y descripcion del modelo
    _name='modulo'
    _description='Módulo del ciclo formativo'
    _rec_name='nombre'

    #Atributos
    nombre=fields.Char('Nombre del módulo',required=True)
    ciclo_id=fields.Many2one('ciclo.formativo',string='Ciclo formativo', required=True)
    profesor_id=fields.Many2one('ciclo.profesor',string='Profesor que imparte')
    alumno_ids=fields.Many2many('ciclo.alumno',string='Alumnos matriculados')
\end{codeblock}

\subsection{Modelo Alumno y Profesor}

\begin{codeblock}
#Definimos modelo Alumno
class Alumno(models.Model):
    #Nombre y descripcion del modelo
    _name='ciclo.alumno'
    _description='Alumno del instituto'
    _rec_name='nombre'

    #Atributos
    nombre=fields.Char('Nombre',required=True)
    modulo_ids=fields.Many2many('modulo',string='Módulos inscritos')

class Profesor(models.Model):
    #Nombre y descripcion del modelo
    _name='ciclo.profesor'
    _description='Profesor del instituto'
    _rec_name='nombre'

    #Atributos
    nombre=fields.Char('Nombre', required=True)
    modulo_ids = fields.One2many('modulo', 'profesor_id', string='Módulos que imparte')    
\end{codeblock}
\newpage
\subsection{Vistas, Acciones y Menús}

Se definen vistas de lista y formulario para cada modelo. Las acciones permiten acceder
a estas vistas y los menús organizan los modelos en la interfaz de Odoo:

\begin{codeblock}
<odoo>
    <!-- Vistas Ciclos -->
    <record id="view_ciclo_formativo_list" model="ir.ui.view">
        <field name="name">ciclo.formativo.list</field>
        <field name="model">ciclo.formativo</field>
        <field name="arch" type="xml">
            <list>
                <field name="nombre"/>
            </list>
        </field>
    </record>

    <record id="view_ciclo_formativo_form" model="ir.ui.view">
        <field name="name">ciclo.formativo.form</field>
        <field name="model">ciclo.formativo</field>
        <field name="arch" type="xml">
            <form>
                <field name="nombre"/>
                <field name="modulo_ids"/>
            </form>
        </field>
    </record>

    <!-- Vistas Módulo -->
    <record id="view_modulo_list" model="ir.ui.view">
        <field name="name">modulo.list</field>
        <field name="model">modulo</field>
        <field name="arch" type="xml">
            <list>
                <field name="nombre"/>
                <field name="ciclo_id"/>
                <field name="profesor_id"/>
            </list>
        </field>
    </record>

    <record id="view_modulo_form" model="ir.ui.view">
        <field name="name">modulo.form</field>
        <field name="model">modulo</field>
        <field name="arch" type="xml">
            <form>
                <field name="nombre"/>
                <field name="ciclo_id"/>
                <field name="profesor_id"/>
                <field name="alumno_ids"/>
            </form>
        </field>
    </record>

    <!-- Vistas Alumno -->
    <record id="view_alumno_list" model="ir.ui.view">
        <field name="name">alumno.list</field>
        <field name="model">ciclo.alumno</field>
        <field name="arch" type="xml">
            <list>
                <field name="nombre"/>
            </list>
        </field>
    </record>

    <record id="view_alumno_form" model="ir.ui.view">
        <field name="name">alumno.form</field>
        <field name="model">ciclo.alumno</field>
        <field name="arch" type="xml">
            <form>
                <field name="nombre"/>
                <field name="modulo_ids"/>
            </form>
        </field>
    </record>

    <!-- Vistas Profesor -->
    <record id="view_profesor_list" model="ir.ui.view">
        <field name="name">profesor.list</field>
        <field name="model">ciclo.profesor</field>
        <field name="arch" type="xml">
            <list>
                <field name="nombre"/>
            </list>
        </field>
    </record>

    <record id="view_profesor_form" model="ir.ui.view">
        <field name="name">profesor.form</field>
        <field name="model">ciclo.profesor</field>
        <field name="arch" type="xml">
            <form>
                <field name="nombre"/>
                <field name="modulo_ids"/>
            </form>
        </field>
    </record>

    <!-- Acciones y Menús -->
    <record id="action_ciclo_formativo" model="ir.actions.act_window">
        <field name="name">Ciclos Formativos</field>
        <field name="res_model">ciclo.formativo</field>
        <field name="view_mode">list,form</field>
    </record>

    <record id="action_modulo" model="ir.actions.act_window">
        <field name="name">Módulos</field>
        <field name="res_model">modulo</field>
        <field name="view_mode">list,form</field>
    </record>

    <record id="action_alumno" model="ir.actions.act_window">
        <field name="name">Alumnos</field>
        <field name="res_model">ciclo.alumno</field>
        <field name="view_mode">list,form</field>
    </record>

    <record id="action_profesor" model="ir.actions.act_window">
        <field name="name">Profesores</field>
        <field name="res_model">ciclo.profesor</field>
        <field name="view_mode">list,form</field>
    </record>

    <!-- Menús -->
    <menuitem id="menu_ciclos_root" name="Ciclos Formativos"/>
    <menuitem id="menu_ciclos" name="Ciclos" parent="menu_ciclos_root" action="action_ciclo_formativo"/>
    <menuitem id="menu_modulos" name="Módulos" parent="menu_ciclos_root" action="action_modulo"/>
    <menuitem id="menu_alumnos" name="Alumnos" parent="menu_ciclos_root" action="action_alumno"/>
    <menuitem id="menu_profesores" name="Profesores" parent="menu_ciclos_root" action="action_profesor"/>
</odoo>


\end{codeblock}

De igual manera, se definen los menús para módulos, alumnos y profesores, facilitando
la navegación dentro del sistema. Además, se configuran grupos de usuario para controlar
los permisos de lectura y escritura, diferenciando entre directores y profesores.

\end{document}
